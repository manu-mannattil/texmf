% vim: ft=tex fdm=marker et sts=2 sw=2
%! TEX program = pdfLaTeX

\documentclass[noamssymb,noamsmath,aspectratio=169]{beamer}

\usepackage[T1]{fontenc}
\usepackage[utf8]{inputenc}
\usepackage[sans]{mathtime}

\usetheme{emem}

\title{Title of Talk}
\author{\texorpdfstring{\textbf{A.~Author}}{A.~Author} \and B.~Author \and C.~Author}
\institute{Untitled Department, Untitled Institute}
\date{1 January 1970}

\begin{document}

\begin{frame}[noframenumbering,plain]
  \titlepage
  \begin{center}
    {\small 1970 Conference on Untitled Subject, Untitled City\\[\baselineskip]}
    {\footnotesize Supported by ABC 123456 (A.A.) and XYZ 123456 (B.B.)}
  \end{center}
\end{frame}

\begin{frame}{Canonical partition function}
Consider $N$ identical particles in 3D.
The canonical partition function $\mathcal{Z}$ is defined as
\begin{equation}
  \mathcal{Z} = \frac{1}{N! h^{3N}}\int d^{N}\mathbf{p}\, d^{N}\mathbf{q}\, e^{-\beta H(\mathbf{p},\mathbf{q})}\,.
\end{equation}
%
If $H(\mathbf{p},\mathbf{q}) = \mathbf{p}^2/2m + U(\mathbf{q})$, one can integrate out the momentum contribution to the partition function and write
\begin{equation}
  \mathcal{Z} = \frac{1}{N! h^{3N}}\left(\frac{2\pi m}{\beta}\right)^{3N/2}\int d^{N}\mathbf{q}\, e^{-\beta U(\mathbf{q})}\,.
\end{equation}
%
Thus, up to multiplicative constants that are \emph{independent} of the coordinates $\mathbf{q}$,
\begin{equation}
  \mathcal{Z} \sim \int d^{N}\mathbf{q}\, e^{-\beta U(\mathbf{q})}\,.
\end{equation}
\end{frame}

\begin{frame}{Remark}

\begin{block}{Remark}
Sample text
\end{block}

\end{frame}

\begin{frame}[noframenumbering,plain]{Backup slide}
  Backup slides should not have page numbers.
\end{frame}

\end{document}
