% vim: ft=tex fdm=marker et sts=2 sw=2

\documentclass[noamssymb,noamsmath,aspectratio=169,9pt]{beamer}

\usetheme{emem}
\usepackage{fourierx}
\usepackage[sans]{fammath}

\title{Title of My Long Winded Talk Goes Here}
\subtitle{arXiv:1970.00000 [cond-mat.soft]}
\author{A. Author\inst{1,2} \and B. Author\inst{1} \and C.  Author\inst{2}}
\institute{\inst{1} School of XXX, Untitled University \and
           \inst{2} School of YYY, Untitled University}
           \titlegraphic{\includegraphics[height=35pt]{example-image.pdf}}
\date{January 1, 1970}
\titlefoot{1970 Conference on Untitled Subject, Untitled City}

\begin{document}

\begin{frame}[noframenumbering,plain]
  \titlepage
\end{frame}

\begin{frame}{Equations}
  A set of \emph{very} famous equations in physics.
  \begin{align*}
    \nabla\cdot\mathbf{E} &= \frac{\rho}{\epsilon_0}\\
    \nabla\cdot\mathbf{B} &= 0\\
    \nabla\times\mathbf{E} &= -\frac{\partial\mathbf{B}}{\partial{t}}\\
    \nabla\times\mathbf{B} &= \mu_0\mathbf{J} + \epsilon_0\mu_0\frac{\partial\mathbf{E}}{\partial{t}}
  \end{align*}
\end{frame}

\begin{frame}{Blocks}

\alert{Normal alert text.} \altalert{Alt alert text.}

\begin{block}{Remark}
Sample text
\end{block}

\end{frame}

\begin{frame}{Definitions \& Examples}
  \begin{definition}
    A \alert{prime number} is a number that has exactly two divisors.
  \end{definition}
  \begin{example}
  \begin{itemize}
    \item 2 is prime (two divisors: 1 and 2).
    \item 3 is prime (two divisors: 1 and 3).
    \item 4 is not prime (\alert{three} divisors: 1, 2, and 4).
  \end{itemize}
  \end{example}
\end{frame}

\begin{frame}{Blocks \& Columns}

  We'll now look at blocks and columns!
  \pause

  \begin{block}{Block Title}
    Block contents.
  \end{block}

  \pause
  \begin{columns}
    \column{.45\textwidth}
    \begin{block}{Small Block 1}
      Block contents.
    \end{block}

    \column{.45\textwidth}
    \begin{block}{Small Block 1}
      Block contents.
    \end{block}
  \end{columns}
\end{frame}

\begin{frame}{Bullet Points}

\begin{itemize}
  \item First bullet point.
    \pause
  \item Second bullet point.
    \pause
  \item Third bullet point.
\end{itemize}
\end{frame}

\begin{frame}[noframenumbering,plain]{Backup slide}
  Backup slides should not have page numbers.
\end{frame}

\end{document}
